\documentclass[../main.tex]{subfiles}
\pagestyle{fancyplain}
\setlength{\headheight}{14mm}%senno latex rognava
\renewcommand{\chaptermark}[1]{\markboth{ #1}{}}
\renewcommand{\sectionmark}[1]{\markright{ #1}}

\lhead[\fancyplain{}{\sffamily\bfseries\thepage}]%
      {\fancyplain{}{\sffamily\bfseries\rightmark}}
\rhead[\fancyplain{}{\sffamily\bfseries\leftmark}]%
      {\fancyplain{}{\sffamily\bfseries\thepage}}
\cfoot{}
\begin{document}

\chapter{Acknowledgment}
\label{cap:aknowledgement}

his paper includes data gathered with the 6.5-m Magellan Telescopes located at Las Campanas Observatory, Chile. 
The STARLIGHT project is supported by the Brazilian agencies CNPq, CAPES and FAPESP and by the France–Brazi CAPES/Cofecub program. 
This research has made use of the NASA/IPAC Ex-
tragalactic Database (NED) which is operated by the Jet Propulsion Laboratory, California Institute of Technology, under contract with the National Aeronautics and Space Administration. 
This study is based on observations made with the NASA/ESA
Hubble Space Telescope and obtained from the Hubble Legacy Archive, which is a collaboration between the Space Telescope Science Institute (STScI/NASA), the Space Telescope European Coordinating Facility (ST-ECF/ESA) and the
Canadian Astronomy Data Centre (CADC/NRC/CSA).
The National Radio Astronomy Observatory is a facility of the National Science Foundation operated under cooperative agreement by Associated Universities, Inc. 
This research has made use of the NASA/IPAC Infrared Science Archive, which is operated by the Jet Propulsion Laboratory, California Institute of Technology, under contract with the National Aeronautics and Space Administration. 
This publication makes use of data products from the Wide-field Infrared Survey Explorer, which is a joint project of the University of California, Los Angeles, and the Jet Propulsion Laboratory/California Institute of Technology, funded by the National Aeronautics and Space Administration.
GMRT is run by the National Centre for Radio Astrophysics of the Tata Institute of Fundamental Research. 
Funding for the Sloan Digital Sky Survey has been provided by the Alfred P. Sloan Foundation and the US Department of Energy Office of Science. The SDSS web site is http://www.sdss.org. SDSS-III is managed by the Astrophysical Research Consortium for the Participating Institutions of the SDSS-III Collaboration including the University of Arizona, the Brazilian Participation Group, Brookhaven National Laboratory, Carnegie Mellon University, University of Florida, the French Participation Group, the German Participation Group, Harvard University, the Instituto de Astrofisica de Canarias, the Michigan State/Notre Dame/JINA Participation Group, Johns Hopkins University, Lawrence Berkeley National Laboratory, Max Planck Institute for Astrophysics,  Max  Planck  Institute  for  Extraterrestrial  Physics,  New  Mexico  State University, University of Portsmouth, Princeton University, the Spanish Participation Group, University of Tokyo, University of Utah, Vanderbilt University, University of Virginia, University of Washington, and Yale University.
This work is based on observations made with the Copernico Telescope of the INAF-Asiago Observatory.













\end{document}