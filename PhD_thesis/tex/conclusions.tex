\documentclass[../main.tex]{subfiles}
\begin{document}

\chapter{Conclusions}
\label{cap:conclusions}

During my PhD I investigated several aspects of the physics of the ENLR in nearby AGN.
I used several techniques and observation at optical and radio wavelengths to have a better insight of the processes characterizing these regions.

In particular my work has been focused on the study of three peculiar objects, IC 5063, NGC 7212 and Mrk 783.
The first two were objects with a known ENLR while the ENLR of Mrk 783 was discovered as a consequence of one of the side projects I was working on during this three year period.

During my PhD I chose to study in details only a few objects instead that a large sample.
The reason for this were two: 1) even though the number of ENLR known is slowly increasing, up until now there are only a hand full of object with an ENLR at low redshift with respect to the number of AGN known, 2) the nature of the ENLR itself make the study of such objects extremely difficult.
The fact that the ENLR is extended and resolved and that there are not two similar ENLRs make it extremely complicate to observe and analyze a large sample of this structure.
Moreover, only a deep and dedicated analysis can discover the secrets beyond each one of this sources.

All the three galaxies I analyzed are characterized by extended radio emission.
While this characteristic was well known for IC 5063 and NGC 7212, I discovered the KSR of Mrk 783 analyzing the data of a recent JVLA survey of NLS1 galaxies (see Ch.\,\ref{cap:paper2}).
This kind of relation between radio emission and ENLR has already been observed several times \citep[e.g.][]{Wilson94,Schmitt03, Schmitt03b} and it is in agreement with the hypothesis that ENLR and relativistic jets (often the source of the radio emission) are connected \citep{Wilson94}.
A property suggesting that this relation might actually be true is the fact that all the objects I examined show hints of the presence of shocks in the interstellar medium, which are often produced by the interaction of the jet with the galactic ISM.
Shocks are more evident in IC 5063 and NGC 7212, where they manifest in the diagnostic diagrams and are required by the models to explain the observed line ratios, while in Mrk 783 their presence is more subtle and uncertain, since only the high temperature of the gas might suggest that shocks are in action.

Another important discovery I made about IC 5063 is that its ionization parameter depends on the velocity of the gas, which is a proof that the ENLR of this object has an hollow bi-conical shape.
This discovery has not been repeated in NGC 7212 and a similar study has not been possible for Mrk 783, because of the low resolution of the spectra and of the low SNR of the spectra around the [\ion{O}{II}]$\lambda\lambda3726,3729$ doublet needed to study the ionization parameter.

On the other hand, the huge size of Mrk 783 ENLR is one of the main results of my work.
With a maximum extension of $\sim 38\,\si{kpc}$ this ENLR is one of the largest ENLR observed at low redshift.
Mrk 783 shows a disturbed morphology and kinematics probably caused by a quite recent merging event.
Since other two galaxies showing similar extensions \citep[UGC 7342 and NGC 5972][]{Keel12} also shows evidences of recent merging, it is possible that the size of the ENLR is connected to the merging event.
However, NGC 7212 that is in interaction with its companion galaxies does not show a particularly large ENLR \citep{Cracco11}, therefore the relation between ENLR and interaction between galaxies deserves a more detailed investigation.

In this work I also discovered the limit of traditional techniques in studying this kind of extended objects.
Since most of the light produced in the ENLR is concentrated on narrow emission lines (in particular [\ion{O}{III}]$\lambda5007$), these structures are difficult to detect in normal broad band images \citep[e.g.][]{Sun18} while long slit spectra are more sensitive to line emission, but they lack the 2D spatial information that are fundamental to characterized the morphology and other properties of ENLRs.
However, while instruments with narrow-band filters are becoming rarer and rarer, new efficient integral field spectrographs are becoming available to the astronomical community opening a new range of possibilities to study optical extended emission in nearby AGN.

On the other hand, I confirmed that extended radio emission can be used as a tracer of extended optical emission and this property should be exploited to search for new ENLRs.
New public high resolution sky surveys such as the VLASS \citep[VLA Sky Survey][]{Hales13}, the LoTSS \citep[LOFAR two meter sky survey][]{Shimwell17}, the GLEAM \citep[GaLactic and Extragalactic All-sky MWA Survey][]{Hurley17}, as well as dedicated surveys will increase the number of known KSR in AGN that can be used to look for optical ENLRs.
A detailed study of these new sample of ENLRs will be fundamental to finally discover how the AGN and its host galaxy interacts and live together.




\end{document}