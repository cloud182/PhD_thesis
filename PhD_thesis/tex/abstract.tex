\documentclass[../main.tex]{subfiles}
\pagestyle{fancyplain}
\setlength{\headheight}{14mm}%senno latex rognava
\renewcommand{\chaptermark}[1]{\markboth{ #1}{}}
\renewcommand{\sectionmark}[1]{\markright{ #1}}

\lhead[\fancyplain{}{\sffamily\bfseries\thepage}]%
      {\fancyplain{}{\sffamily\bfseries\rightmark}}
\rhead[\fancyplain{}{\sffamily\bfseries\leftmark}]%
      {\fancyplain{}{\sffamily\bfseries\thepage}}
\cfoot{}
\begin{document}

\chapter{Abstract}
\label{sec:abstract}
%\addcontentsline{toc}{chapter}{Abstract}

L'\emph{extended narrow-line region} (ENLR) \`e una delle strutture tipiche dei nuclei galattici attivi (AGN) pi\`u interessanti e meno studiate.
\`E una struttura di gas, ionizzato dalla radiazione prodotta nelle regioni pi\`u interne dell'AGN.
Il suo spettro e simile a quello di un'altra struttura tipica di questi oggetti, la \emph{narrow-line region} (NLR).
La differenza principale tra le due strutture \`e che l'ENLR  ha una dimensione che supera il kiloparsec, alcune superano i $20\,\si{kpc}$, mentre la NLR \`e concentrata entro qualche centinaia di parsec dal nucleo.

In questo lavoro ho usato spettri e immagini (nella banda ottica e nel radio), per esaminare le propriet\`a di alcuni di questi oggetti caratterizzati dalla presenza di queste strutture particolari.
Per prima cosa, ho usato spettri ad alta risoluzione e modelli di foto-ionizzazione e shocks per studiare le propriet\`a fisiche del gas in due oggetti con una nota ENLR: NGC 7212 e IC 5063.
La risoluzione spettrale ha consentito di risolvere tutte le righe osservate, il cui profilo \`e risultato essere caratterizzato da picchi multipli e assimmetrie.
Inoltre ho potuto analizzare le propriet\`a del gas in funzione della velocit\`a dello stesso.
In questo modo ho scoperto che, in entrambi gli oggetti, gli schock potrebbero contribuire in maniera consistente alla ionizzazione del gas che si muove ad alte velocit\`a.
Questa propriet\`a, combinata con la cinematica complessa che si evince dal profilo delle righe, potrebbe essere la prova dell'interazione tra i jet dell'AGN e il mezzo interstellare (ISM).

In seguito, ho iniziato a studiare la relazione tra l'ENLR e l'emissione radio estesa presente in molti AGN.
Per prima cosa, ho cercato nuovi oggetti con emissione radio estesa in una nuova survey ottenuta con il VLA at $5\,\si{GHz}$ di un campione di AGN appartenenti alla classe particolare delle \emph{narrow-line Seyfert 1 galaxies} (NLS1s).
Il primo risultato della survey, e probabilmente anche il pi\`u interessante, \`e stato la scoperta di un'emissione estesa con le caratteristiche di un \emph{relic}, in Mrk 783, una NLS1 dell'universo vicino.
Nuove immagini e spettri, acquisiti dopo la scoperta dell'emissione radio, hanno rivelato la presenza di una ENLR.
La struttura di gas ionizzato \`e una dell pi\`u estese scoperte finora, si pu\`o tracciare l'emissione fino ad una distanza di $38\,\si{kpc}$ dal nucleo, ed \`e allineata con la parte pi\`u estesa dell'emissione radio.
Inoltre, la galassia ospite mostra segni di un recente merging con una compagna e potrebbe essere nelle prime fasi dell'interazione con una seconda sorgente vicina.

\clearpage

\chapter*{Abstract}

The extended narrow-line region (ENLR) is one of the most interesting and less studied structures typical of active galactic nuclei (AGN).
It is made of gas, ionized by the radiation produced in the inner region of the AGN.
Its spectral properties are similar to those of the narrow-line region (NLR), but the extension of ENLR is usually larger than $1\,\si{kpc}$ (there are objects with ENLR larger than $20\,\si{kpc}$) while the NLR is often concentrated in a radius of some hundreds parsecs from the AGN.  

In this work I used optical spectra and radio and optical images to investigate the properties of some of these peculiar structures.
In particular, I firstly used high resolution spectra and models combining photo-ionization and shocks to study the physical conditions of the gas in the ENLR of two nearby sources, NGC 7212 and IC 5063.
The spectral resolution of the data allowed to resolve all the observed lines, which show a complex profile characterized by multiple peaks, asymmetries and bumps.
Moreover, it allowed me to investigate the behavior of the gas properties as a function of its velocity.
In this way I discovered that, in both AGN, shocks might give an important contribution to the ionization of the gas moving at high velocities. 
This property, together with the complex kinematics of the gas, might be the result of the interaction between the AGN jets and the interstellar medium (ISM) of the galaxies.

Then, I started investigating the relation between extended optical emission and extended radio emission.
For this reason, I looked for new extended radio structures in a VLA survey at $5\,\si{GHz}$ of a sample of AGN belonging to the peculiar class of narrow-line Seyfert 1 galaxies (NLS1s). 
The first and most interesting result of this survey was the discovery of a very interesting radio emission, probably a relic, in Mrk 783, a nearby NLS1.
An optical follow-up of the source revealed the presence of an ENLR.
The structure is one of the largest discovered so far, with a maximum extension of $38\,\si{kpc}$ just aligned with the most extended part of the radio emission.
The host galaxy also shows signs of a recent merging with a companion, and it might be in the first stages of interaction with another nearby source.


\end{document}