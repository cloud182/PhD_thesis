\documentclass[../thesis.tex]{subfiles}
\begin{document}

\chapter{Introduction}
\label{cap:intro}
%\addcontentsline{toc}{chapter}{Introduction}

Active galactic nuclei (AGN) are known to be some of the most powerful emitters of the whole universe.
The AGN class gather together objects showing a large variety of properties but that are all powered by the same engine: accretion of matter on a supermassive black hole.
Even though the region where the powering processes take place is small compared to the dimension of a galaxy, the presence of an AGN can affect the properties of its host in a great variety of ways.

One of the most spectacular way in which an AGN can affects its host galaxy is the creation of extended emission regions which can be as large as the whole galaxy.
The properties of these structures change depending on the part of the electromagnetic spectrum where they manifest themselves but, nevertheless, they all seem to be connected.
Unfortunately extended emission regions are not as common as other AGN typical structures and they are observed only on a small number of objects.

The main purpose of this thesis is to study the properties of such extended emission regions at optical and radio wavelengths.
The final goal is to shed some more light in the processes that take place in these regions, which can be fundamental in the so called AGN feedback, the effects that an AGN produces on its host.
In the first chapter of this work I will summarize the properties of active galactic nuclei, focusing in particular on extended emission regions at optical and radio wavelength.
Since it is important to know the characteristic of the radiation we receive from this structures, in the second chapter I will shortly revise the dominant AGN emission processes in the regions of the electromagnetic spectrum I am interested in.

The main body of the work will be described in the following three chapter where I am going to report the result of the research I have been carrying on in the last three year.
Chapter \ref{cap:paper1} will report the results of the study of high resolution optical spectra of two well known galaxies with extended emission line region.
Chapter \ref{cap:paper2} will instead report the discovery of a kiloparsec scale radio emission region, one of the few of its kind, in a narrow-line Seyfert 1 galaxy, a peculiar class of AGN whose nature is still debated and that is rising more and more questions after the detection of $\gamma$-ray emission from one of them.
Chapter \ref{cap:paper3} will present the analysis of new optical and radio data on this object which are rising even more questions with respect to the answer they are giving.

Finally, in the last chapter I will draw some conclusion from the overall results of my work.

\biblio
\end{document}