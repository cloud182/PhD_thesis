%__________________________________________________________________________________________________
%% DOCUMENT SETUP
%__________________________________________________________________________________________________
\documentclass[11pt, a4paper, twoside, openright]{book}   
\usepackage[binding=1cm]{layaureo}   %automatically defines margins for binding           

% %PACKAGES

\usepackage{LukeThesis}                             %Import Thesis base style
%\usepackage{subcaption}
%\usepackage[a-1b]{pdfx}                 %pdfA-1b file
\usepackage{amsmath}
\usepackage[nohints]{minitoc}
\usepackage{algorithm}% a quanto pare serve per generare il prefrontmatter
\usepackage{lipsum}
\usepackage{listings}
\usepackage{subfig}
\usepackage{graphicx}
\graphicspath{{PhD_thesis/images/}{PhD_thesis/images/paper1}{PhD_thesis/images/paper2}}
\usepackage{frontespizio} % creare frontespizio
\usepackage{aas_macros} %riconoscere abbreviazioni giornali
\usepackage{natbib} %bibliografia
%\usepackage{bibentry}
\usepackage{subfiles} %compilare i capitoli in file a parte
\usepackage[flushleft]{threeparttable} %note sotto tabelle
\usepackage{mathtools}
\usepackage{epigraph} %sottotitoli sotto capitolo
\usepackage{mfirstuc}
\usepackage{pdflscape}
\usepackage{siunitx} %unita' di misura
\DeclareSIUnit\msun{M\ensuremath{_\odot}}
\DeclareSIUnit\zmet{Z\ensuremath{_\odot}}
\DeclareSIUnit\ergs{\erg\per\second}
\DeclareSIUnit\kms{km~s^{-1}}
\renewcommand{\vec}[1]{\mathbf{#1}}
\DeclarePairedDelimiter{\abs}{\lvert}{\rvert}
\DeclarePairedDelimiter{\norma}{\lVert}{\rVert}

\usepackage[nottoc,numbib]{tocbibind}
\usepackage{chngcntr}
\usepackage{hyperref} %Da sistemare, andra' in conflitto con la certificazione pdfa 

%__________________________________________________________________________________________________
% %SECTION NUMBERING SETUP
%__________________________________________________________________________________________________
\setcounter{tocdepth}{2}  %2 adds sections up to subsections
\setcounter{secnumdepth}{3}  %Subsubsections get a number when this is 3

%__________________________________________________________________________________________________
% %Dedicated commands
%__________________________________________________________________________________________________

\newcommand{\ion}[2]{\capitalisewords{#1}\,\uppercase{#2}}
\makeatletter
\def\blfootnote{\xdef\@thefnmark{}\@footnotetext}
\makeatother
\maxdeadcycles=200
\newcommand{\Ha}{\rm H\alpha }
\newcommand{\Hb}{\rm H\beta }
\newcommand{\kms}{km s^{-1}}
\newcommand{\ergs}{erg s^{-1}}
\def\xr#1{\parindent=0.0cm\han indent=1cm\han after=1\indent#1\par}
\def\la{\raise.5ex\hbox{$<$}\kern-.8em\lower 1mm\hbox{$\sim$}}
\def\ma{\raise.5ex\hbox{$>$}\kern-.8em\lower 1mm\hbox{$\sim$}}
\def\ea{\it et al. \rm}
\def\am{$^{\prime}$\ }
\def\as{$^{\prime\prime}$\ }
\def\msol{M$_{\odot}$ }
\def\Lsol{L$_{\odot}$ }
%\def\kms{$\rm km\, s^{-1}$}
\def\cm3{$\rm cm^{-3}$}
\def\Ts{$\rm T_{*}$~}
\def\Vs{$\rm V_{s}$~}
\def\n0{$\rm n_{0}$}
\def\B0{$\rm B_{0}$}
\def\ne{$\rm n_{e}$~}
\def\Ne{$\rm N_{e}$}
\def\Te{$\rm T_{e}$}
\def\Tgr{$\rm T_{gr}$}
\def\Td{$\rm T_{d}$}
\def\Tgas{$\rm T_{gas}$}
\def\Ec{$\rm E_{c}$}
\def\Fn{$\rm F_{n}$}
\def\Fh{$\rm F_{h}$~}
\def\Fnu{$F_{\nu}$~}
\def\erg{$\rm erg\, cm^{-2}\, s^{-1}$}
\def\mum{$\mu$m~}
\def\LIR{L$_{IR}$~}
\def\L12{L$_{12\mu m}$~}
\def\F12{F$_{12\mu m}$~}
\def\agr{a$_{gr}$}
%\def\Hb{H${\beta}$~}
%\def\Ha{H${\alpha}$~}
\def\Hg{H${\gamma}$~}
\def\Haa{H${\alpha}_{calc}$~}
\def\Ly{Ly$\alpha$~}
\def\La{L$_{H\alpha}$~}
\def\Moy{M$_{\odot}$ yr$^{-1}$}
\def\Tef{T$_{eff}$}

%% footnote without a number

%__________________________________________________________________________________________________
%% THESIS STRUCTURE  (Modifiy to include more chapters etc)
%__________________________________________________________________________________________________
\def\biblio{\bibliographystyle{mnras}\bibliography{bibliografia_def}} %to have bibliography in the chapter
\begin{document}
\def\biblio{} %undefining the previous command

%------------------------
%Pre-frontmatter material
%------------------------

\begin{frontespizio}
\Logo[3.0cm]{images/logo_unipd.png}
\Universita{Padova}
\Dipartimento{Fisica e Astronomia Galileo Galilei}
\Scuola{Sede Amministrativa: Universit\`a degli Studi di Padova}
\Titoletto{Corso di dottorato di ricerca in Astronomia, ciclo XXXI}
\Titolo{The physics of the extended narrow-line region in active galactic nuclei}
\NCandidato{Dottorando}
\Candidato{Enrico Congiu}
\NRelatore{Coordinatore}{Relatori}
\Relatore{Chiar.mo Prof. Giampaolo Piotto}
\NCorrelatore{Supervisore}{Supervisori}
\Correlatore{Dott. Stefano Ciroi}
\Annoaccademico{2017/2018}
\Rientro{1.5cm}
\Punteggiatura{}
\end{frontespizio}

%--------------------
%Frontmatter material
%--------------------

\frontmatter
\pagenumbering{roman}                               
\let\mtcontentsname\contentsname
\renewcommand\contentsname{\MakeUppercase\mtcontentsname}

\tableofcontents %io ho messo solo questo, vedi quello sotto messo da luke

\newpage

\subfile{tex/abstract}
\subfile{tex/introduction}
%-------------
% Main content
%-------------
\mainmatter


\subfile{tex/chapter1} %Extended narrow-line regions in AGN
\subfile{tex/chapter2} %Emission processes in AGN
\subfile{tex/chapter3} %High-res. spectroscopy of ENLRs
\subfile{tex/chapter4} %KSE in the NLS1 Mrk 783
\subfile{tex/chapter5} %Optical analysis of Mrk 783


%-----------
% Backmatter
%-----------
\backmatter
\subfile{tex/conclusions}
\bibliographystyle{mnras}
\bibliography{bibliografia_def}
\appendix
\subfile{tex/appendix1}  %figure paper 1\\
\subfile{tex/appendix2}  %tabelle paper 1
\subfile{tex/aknowledgement}

 


\end{document}
% % % EOF % % %